\documentclass[twocolumn,a4paper]{article}
\usepackage{xcolor,
  graphicx,
  multicol,
  listings,savetrees}
\setlength\columnseprule{.5pt}
\setlength\columnsep{2em}
% \usepackage[a4paper,
%       scale={0.92,1.03},
%       includeheadfoot
%       ]{geometry}
\usepackage{palatino}

\definecolor{fontys}{rgb}{.3,0,.3}
\lstset{language=Java,basicstyle={\ttfamily}}
\lstset{basicstyle={\scriptsize},language=java,frame=shadowbox,rulesepcolor=\color{fontys}}

\providecommand\Code[1]{{\color{fontys}\ttfamily#1}}

\title{PRO2/SEN1 TDD List}
\author{Pieter~van~den~Hombergh}
\begin{document}
\maketitle

\section*{Test driven development of two SimpleList implementations}

In this exercise, you will develop two implementations of a SimpleList
interface. The approach is very similar to the Stack task in SEN1.

The task counts for JAVA2 and SEN1.

Work test driven. Each time you have implemented a test, commit. Each
time you implemented the business logic of a requirement, commit.

The interface is well\footnote{the javadoc in the project zip file} documented. Note that the interface it selves
extends \lstinline{Iterable<E>}, which needs some extra attention.
Implementing Iterable, allows the user of objects of such class to use
the \textit{for-each} for loop type.

\lstinputlisting[caption={SimpleList interface}]{code/SimpleList.java}

\subsection*{Implementing an Iterator}
Implementing \Code{Iterable} means that your class can produce an
\textit{implementation} of \Code{Iterator}, which is another interface.

Iterator declares three methods, of which only two need an actual
implementation. It is okay to leave the \Code{remove()} unimplemented
in the sense that it may throw a
\Code{UnsupportedOperationException}. In Java 8 \Code{remove()} is a so called
default method that does exactly that.

Implementing an iterator is typically done using an inner class.
A instance of said inner class is initialized with members that are able to
remember a \textbf{position} of the ``next'' element inside the
collection (list in this case). The method \Code{next()} can then
fetch the next element AND advance the position and return the fetched
element. 

For a linked list implementation, the \textit{position} is simply another link,
initialized appropriately. For an array list implementation it is
simply an index into the array.

Advancing the position is
\begin{itemize}
\item increment the position index for the array variant
\item move the position to position.next.
\end{itemize}

Not that \Code{hasNext()} should return true, when a following call of
\Code{next()} can produce an element and false when the position
reached the end of the list.

\subsection*{Sorting a  list}
To be able to sort lists of elements, you need to compare those
elements. A comparator is what you need. In Java \lstinline{Comparator<T>} is another
Interface. A comparator compares two
elements, say $a$ and $b$, and produces the following outcomes:
\begin{description}%[srt]
\item[NEGATIVE] Some int value less then zero, saying $a < b$.
\item[ZERO] int 0 which says the elements compare equal as far as the comparator
  is concerned.
\item[POSITIVE] some value int value greater then zero, saying $a > b $.
\end{description}


Classes that themselves implement \lstinline{Comparable<T>} provide a
method \lstinline{int compareTo(T other)}, which produces the result
expected by a comparator, when \textit{this} is $a$ and $b$ is
\textit{other}.
This leads to the following simple implementation of a Comparator:

\lstinputlisting[caption={Comparator for Comparables}]{code/Comparator.java}

\subsection*{Hints for the Linked list implementation}
It appears to be easier to use dummy links for head and tail of the
list.
Head.next is the first element; head.next.payload is the content of
that first link; tail is the element where you have to
add new elements to the list.
Other solutions are possible.

\subsection*{Deliverables}
\begin{itemize}
\item A Netbeans project containing all java files. Check in your
  solution to the appropriate directory named 'tddlist' under trunk.
\end{itemize}
\end{document}
